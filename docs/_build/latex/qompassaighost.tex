%% Generated by Sphinx.
\def\sphinxdocclass{report}
\documentclass[letterpaper,10pt,english]{sphinxmanual}
\ifdefined\pdfpxdimen
   \let\sphinxpxdimen\pdfpxdimen\else\newdimen\sphinxpxdimen
\fi \sphinxpxdimen=.75bp\relax
\ifdefined\pdfimageresolution
    \pdfimageresolution= \numexpr \dimexpr1in\relax/\sphinxpxdimen\relax
\fi
%% let collapsible pdf bookmarks panel have high depth per default
\PassOptionsToPackage{bookmarksdepth=5}{hyperref}

\PassOptionsToPackage{booktabs}{sphinx}
\PassOptionsToPackage{colorrows}{sphinx}

\PassOptionsToPackage{warn}{textcomp}
\usepackage[utf8]{inputenc}
\ifdefined\DeclareUnicodeCharacter
% support both utf8 and utf8x syntaxes
  \ifdefined\DeclareUnicodeCharacterAsOptional
    \def\sphinxDUC#1{\DeclareUnicodeCharacter{"#1}}
  \else
    \let\sphinxDUC\DeclareUnicodeCharacter
  \fi
  \sphinxDUC{00A0}{\nobreakspace}
  \sphinxDUC{2500}{\sphinxunichar{2500}}
  \sphinxDUC{2502}{\sphinxunichar{2502}}
  \sphinxDUC{2514}{\sphinxunichar{2514}}
  \sphinxDUC{251C}{\sphinxunichar{251C}}
  \sphinxDUC{2572}{\textbackslash}
\fi
\usepackage{cmap}
\usepackage[T1]{fontenc}
\usepackage{amsmath,amssymb,amstext}
\usepackage{babel}



\usepackage{tgtermes}
\usepackage{tgheros}
\renewcommand{\ttdefault}{txtt}



\usepackage[Bjarne]{fncychap}
\usepackage{sphinx}

\fvset{fontsize=auto}
\usepackage{geometry}


% Include hyperref last.
\usepackage{hyperref}
% Fix anchor placement for figures with captions.
\usepackage{hypcap}% it must be loaded after hyperref.
% Set up styles of URL: it should be placed after hyperref.
\urlstyle{same}


\usepackage{sphinxmessages}
\setcounter{tocdepth}{1}



\title{Qompass AI Ghost}
\date{Jul 30, 2025}
\release{}
\author{Qompass AI}
\newcommand{\sphinxlogo}{\vbox{}}
\renewcommand{\releasename}{}
\makeindex
\begin{document}

\ifdefined\shorthandoff
  \ifnum\catcode`\=\string=\active\shorthandoff{=}\fi
  \ifnum\catcode`\"=\active\shorthandoff{"}\fi
\fi

\pagestyle{empty}
\sphinxmaketitle
\pagestyle{plain}
\sphinxtableofcontents
\pagestyle{normal}
\phantomsection\label{\detokenize{index::doc}}


\noindent\sphinxincludegraphics[width=400\sphinxpxdimen]{{ghost}.svg}

\sphinxstepscope


\chapter{Qompass AI Quickstart}
\label{\detokenize{quickstart:qompass-ai-quickstart}}\label{\detokenize{quickstart::doc}}
\sphinxAtStartPar
Mail servers can be a tricky thing to set up. This guide is supposed to
run you through the most important steps to achieve a 10/10 score on
\sphinxurl{https://mail-tester.com}.

\sphinxAtStartPar
What you need is:
\begin{itemize}
\item {} 
\sphinxAtStartPar
a server running NixOS with a public IP

\item {} 
\sphinxAtStartPar
a domain name.

\end{itemize}

\begin{sphinxadmonition}{note}{Note:}
\sphinxAtStartPar
In the following, we consider a server with the public IP \sphinxcode{\sphinxupquote{1.2.3.4}}
and the domain \sphinxcode{\sphinxupquote{example.com}}.
\end{sphinxadmonition}

\sphinxAtStartPar
First, we will set the minimum DNS configuration to be able to deploy
an up and running mail server. Once the server is deployed, we could
then set all DNS entries required to send and receive mails on this
server.


\section{Setup DNS A/AAAA records for server}
\label{\detokenize{quickstart:setup-dns-a-aaaa-records-for-server}}
\sphinxAtStartPar
Add DNS records to the domain \sphinxcode{\sphinxupquote{example.com}} with the following
entries


\begin{savenotes}\sphinxattablestart
\sphinxthistablewithglobalstyle
\centering
\begin{tabulary}{\linewidth}[t]{TTTT}
\sphinxtoprule
\sphinxstyletheadfamily 
\sphinxAtStartPar
Name (Subdomain)
&\sphinxstyletheadfamily 
\sphinxAtStartPar
TTL
&\sphinxstyletheadfamily 
\sphinxAtStartPar
Type
&\sphinxstyletheadfamily 
\sphinxAtStartPar
Value
\\
\sphinxmidrule
\sphinxtableatstartofbodyhook
\sphinxAtStartPar
\sphinxcode{\sphinxupquote{mail.example.com}}
&
\sphinxAtStartPar
10800
&
\sphinxAtStartPar
A
&
\sphinxAtStartPar
\sphinxcode{\sphinxupquote{1.2.3.4}}
\\
\sphinxhline
\sphinxAtStartPar
\sphinxcode{\sphinxupquote{mail.example.com}}
&
\sphinxAtStartPar
10800
&
\sphinxAtStartPar
AAAA
&
\sphinxAtStartPar
\sphinxcode{\sphinxupquote{2001::1}}
\\
\sphinxbottomrule
\end{tabulary}
\sphinxtableafterendhook\par
\sphinxattableend\end{savenotes}

\sphinxAtStartPar
If your server does not have an IPv6 address, you must skip the \sphinxtitleref{AAAA} record.

\sphinxAtStartPar
You can check this with

\begin{sphinxVerbatim}[commandchars=\\\{\}]
\PYGZdl{} nix\PYGZhy{}shell \PYGZhy{}p bind \PYGZhy{}\PYGZhy{}command \PYGZdq{}host \PYGZhy{}t A mail.example.com\PYGZdq{}
mail.example.com has address 1.2.3.4

\PYGZdl{} nix\PYGZhy{}shell \PYGZhy{}p bind \PYGZhy{}\PYGZhy{}command \PYGZdq{}host \PYGZhy{}t AAAA mail.example.com\PYGZdq{}
mail.example.com has address 2001::1
\end{sphinxVerbatim}

\sphinxAtStartPar
Note that it can take a while until a DNS entry is propagated. This
DNS entry is required for the Let\textquotesingle{}s Encrypt certificate generation
(which is used in the below configuration example).


\section{Setup the server}
\label{\detokenize{quickstart:setup-the-server}}
\sphinxAtStartPar
The following describes a server setup that is fairly complete. Even
though there are more possible options (see the \sphinxhref{options.html}{Qompass AI Ghost
options documentation}), these should be the most
common ones.

\begin{sphinxVerbatim}[commandchars=\\\{\}]
\PYG{p}{\PYGZob{}} config\PYG{p}{,} pkgs\PYG{p}{,} \PYG{o}{.}\PYG{o}{.}\PYG{o}{.} \PYG{p}{\PYGZcb{}}\PYG{p}{:} \PYG{p}{\PYGZob{}}
  \PYG{l+s+ss}{imports} \PYG{o}{=} \PYG{p}{[}
    \PYG{p}{(}\PYG{n+nb}{builtins}\PYG{o}{.}fetchTarball \PYG{p}{\PYGZob{}}
      \PYG{c+c1}{\PYGZsh{} Pick a release version you are interested in and set its hash, e.g.}
      \PYG{l+s+ss}{url} \PYG{o}{=} \PYG{l+s+s2}{\PYGZdq{}}\PYG{l+s+s2}{https://github.com/qompassai/ghost/\PYGZhy{}/archive/nixos\PYGZhy{}25.05/ghost\PYGZhy{}25.05.tar.gz}\PYG{l+s+s2}{\PYGZdq{}}\PYG{p}{;}
      \PYG{c+c1}{\PYGZsh{} To get the sha256 of the ghost tarball, we can use the nix\PYGZhy{}prefetch\PYGZhy{}url command:}
      \PYG{c+c1}{\PYGZsh{} release=\PYGZdq{}nixos\PYGZhy{}25.05\PYGZdq{}; nix\PYGZhy{}prefetch\PYGZhy{}url \PYGZdq{}https://github.com/qompassai/ghost/\PYGZhy{}/archive/\PYGZdl{}\PYGZob{}release\PYGZcb{}/ghost\PYGZhy{}\PYGZdl{}\PYGZob{}release\PYGZcb{}.tar.gz\PYGZdq{} \PYGZhy{}\PYGZhy{}unpack}
      \PYG{l+s+ss}{sha256} \PYG{o}{=} \PYG{l+s+s2}{\PYGZdq{}}\PYG{l+s+s2}{0000000000000000000000000000000000000000000000000000}\PYG{l+s+s2}{\PYGZdq{}}\PYG{p}{;}
    \PYG{p}{\PYGZcb{}}\PYG{p}{)}
  \PYG{p}{]}\PYG{p}{;}

  \PYG{l+s+ss}{ghost} \PYG{o}{=} \PYG{p}{\PYGZob{}}
    \PYG{l+s+ss}{enable} \PYG{o}{=} \PYG{n+no}{true}\PYG{p}{;}
    \PYG{l+s+ss}{fqdn} \PYG{o}{=} \PYG{l+s+s2}{\PYGZdq{}}\PYG{l+s+s2}{mail.example.com}\PYG{l+s+s2}{\PYGZdq{}}\PYG{p}{;}
    \PYG{l+s+ss}{domains} \PYG{o}{=} \PYG{p}{[} \PYG{l+s+s2}{\PYGZdq{}}\PYG{l+s+s2}{example.com}\PYG{l+s+s2}{\PYGZdq{}} \PYG{p}{]}\PYG{p}{;}

    \PYG{c+c1}{\PYGZsh{} A list of all login accounts. To create the password hashes, use}
    \PYG{c+c1}{\PYGZsh{} nix\PYGZhy{}shell \PYGZhy{}p mkpasswd \PYGZhy{}\PYGZhy{}run \PYGZsq{}mkpasswd \PYGZhy{}sm bcrypt\PYGZsq{}}
    \PYG{l+s+ss}{loginAccounts} \PYG{o}{=} \PYG{p}{\PYGZob{}}
      \PYG{l+s+s2}{\PYGZdq{}}\PYG{l+s+s2}{user1@example.com}\PYG{l+s+s2}{\PYGZdq{}} \PYG{o}{=} \PYG{p}{\PYGZob{}}
        \PYG{l+s+ss}{hashedPasswordFile} \PYG{o}{=} \PYG{l+s+s2}{\PYGZdq{}}\PYG{l+s+s2}{/a/file/containing/a/hashed/password}\PYG{l+s+s2}{\PYGZdq{}}\PYG{p}{;}
        \PYG{l+s+ss}{aliases} \PYG{o}{=} \PYG{p}{[}\PYG{l+s+s2}{\PYGZdq{}}\PYG{l+s+s2}{postmaster@example.com}\PYG{l+s+s2}{\PYGZdq{}}\PYG{p}{]}\PYG{p}{;}
      \PYG{p}{\PYGZcb{}}\PYG{p}{;}
      \PYG{l+s+s2}{\PYGZdq{}}\PYG{l+s+s2}{user2@example.com}\PYG{l+s+s2}{\PYGZdq{}} \PYG{o}{=} \PYG{p}{\PYGZob{}} \PYG{o}{.}\PYG{o}{.}\PYG{o}{.} \PYG{p}{\PYGZcb{}}\PYG{p}{;}
    \PYG{p}{\PYGZcb{}}\PYG{p}{;}

    \PYG{c+c1}{\PYGZsh{} Use Let\PYGZsq{}s Encrypt certificates. Note that this needs to set up a stripped}
    \PYG{c+c1}{\PYGZsh{} down nginx and opens port 8080.}
    \PYG{l+s+ss}{certificateScheme} \PYG{o}{=} \PYG{l+s+s2}{\PYGZdq{}}\PYG{l+s+s2}{acme\PYGZhy{}nginx}\PYG{l+s+s2}{\PYGZdq{}}\PYG{p}{;}
  \PYG{p}{\PYGZcb{}}\PYG{p}{;}
  security\PYG{o}{.}acme\PYG{o}{.}\PYG{l+s+ss}{acceptTerms} \PYG{o}{=} \PYG{n+no}{true}\PYG{p}{;}
  security\PYG{o}{.}acme\PYG{o}{.}defaults\PYG{o}{.}\PYG{l+s+ss}{email} \PYG{o}{=} \PYG{l+s+s2}{\PYGZdq{}}\PYG{l+s+s2}{security@example.com}\PYG{l+s+s2}{\PYGZdq{}}\PYG{p}{;}
\PYG{p}{\PYGZcb{}}
\end{sphinxVerbatim}

\sphinxAtStartPar
After a \sphinxcode{\sphinxupquote{nixos\sphinxhyphen{}rebuild switch}} your server should be running all
mail components.


\section{Setup all other DNS requirements}
\label{\detokenize{quickstart:setup-all-other-dns-requirements}}

\subsection{Set rDNS (reverse DNS) entry for server}
\label{\detokenize{quickstart:set-rdns-reverse-dns-entry-for-server}}
\sphinxAtStartPar
Wherever you have rented your server, you should be able to set reverse
DNS entries for the IP’s you own:
\begin{itemize}
\item {} 
\sphinxAtStartPar
Add an entry resolving IPv4 address \sphinxcode{\sphinxupquote{1.2.3.4}} to \sphinxcode{\sphinxupquote{mail.example.com}}.

\item {} 
\sphinxAtStartPar
Add an entry resolving IPv6 \sphinxcode{\sphinxupquote{2001::1}} to \sphinxcode{\sphinxupquote{mail.example.com}}. Again, this
must be skipped if your server does not have an IPv6 address.

\end{itemize}

\begin{sphinxadmonition}{warning}{Warning:}
\sphinxAtStartPar
We don\textquotesingle{}t recommend setting up a mail server if you are not able to
set a reverse DNS on your public IP because sent emails would be
mostly marked as spam. Note that many residential ISP providers
don\textquotesingle{}t allow you to set a reverse DNS entry.
\end{sphinxadmonition}

\sphinxAtStartPar
You can check this with

\begin{sphinxVerbatim}[commandchars=\\\{\}]
\PYGZdl{} nix\PYGZhy{}shell \PYGZhy{}p bind \PYGZhy{}\PYGZhy{}command \PYGZdq{}host 1.2.3.4\PYGZdq{}
4.3.2.1.in\PYGZhy{}addr.arpa domain name pointer mail.example.com.

\PYGZdl{} nix\PYGZhy{}shell \PYGZhy{}p bind \PYGZhy{}\PYGZhy{}command \PYGZdq{}host 2001::1\PYGZdq{}
1.0.0.0.0.0.0.0.0.0.0.0.0.0.0.0.0.0.0.0.0.0.0.0.0.0.0.0.1.0.0.2.ip6.arpa domain name pointer mail.example.com.
\end{sphinxVerbatim}

\sphinxAtStartPar
Note that it can take a while until a DNS entry is propagated.


\subsection{Set a \sphinxstyleliteralintitle{\sphinxupquote{MX}} record}
\label{\detokenize{quickstart:set-a-mx-record}}
\sphinxAtStartPar
Add a \sphinxcode{\sphinxupquote{MX}} record to the domain \sphinxcode{\sphinxupquote{example.com}}.


\begin{savenotes}\sphinxattablestart
\sphinxthistablewithglobalstyle
\centering
\begin{tabulary}{\linewidth}[t]{TTTT}
\sphinxtoprule
\sphinxstyletheadfamily 
\sphinxAtStartPar
Name (Subdomain)
&\sphinxstyletheadfamily 
\sphinxAtStartPar
Type
&\sphinxstyletheadfamily 
\sphinxAtStartPar
Priority
&\sphinxstyletheadfamily 
\sphinxAtStartPar
Value
\\
\sphinxmidrule
\sphinxtableatstartofbodyhook
\sphinxAtStartPar
example.com
&
\sphinxAtStartPar
MX
&
\sphinxAtStartPar
10
&
\sphinxAtStartPar
mail.example.com
\\
\sphinxbottomrule
\end{tabulary}
\sphinxtableafterendhook\par
\sphinxattableend\end{savenotes}

\sphinxAtStartPar
You can check this with

\begin{sphinxVerbatim}[commandchars=\\\{\}]
\PYGZdl{} nix\PYGZhy{}shell \PYGZhy{}p bind \PYGZhy{}\PYGZhy{}command \PYGZdq{}host \PYGZhy{}t mx example.com\PYGZdq{}
example.com mail is handled by 10 mail.example.com.
\end{sphinxVerbatim}

\sphinxAtStartPar
Note that it can take a while until a DNS entry is propagated.


\subsection{Set a \sphinxstyleliteralintitle{\sphinxupquote{SPF}} record}
\label{\detokenize{quickstart:set-a-spf-record}}
\sphinxAtStartPar
Add a \sphinxhref{https://en.wikipedia.org/wiki/Sender\_Policy\_Framework}{SPF}
record to the domain \sphinxcode{\sphinxupquote{example.com}}.


\begin{savenotes}\sphinxattablestart
\sphinxthistablewithglobalstyle
\centering
\begin{tabulary}{\linewidth}[t]{TTTT}
\sphinxtoprule
\sphinxstyletheadfamily 
\sphinxAtStartPar
Name (Subdomain)
&\sphinxstyletheadfamily 
\sphinxAtStartPar
TTL
&\sphinxstyletheadfamily 
\sphinxAtStartPar
Type
&\sphinxstyletheadfamily 
\sphinxAtStartPar
Value
\\
\sphinxmidrule
\sphinxtableatstartofbodyhook
\sphinxAtStartPar
example.com
&
\sphinxAtStartPar
10800
&
\sphinxAtStartPar
TXT
&
\sphinxAtStartPar
\sphinxtitleref{v=spf1 a:mail.example.com \sphinxhyphen{}all}
\\
\sphinxbottomrule
\end{tabulary}
\sphinxtableafterendhook\par
\sphinxattableend\end{savenotes}

\sphinxAtStartPar
You can check this with

\begin{sphinxVerbatim}[commandchars=\\\{\}]
\PYGZdl{} nix\PYGZhy{}shell \PYGZhy{}p bind \PYGZhy{}\PYGZhy{}command \PYGZdq{}host \PYGZhy{}t TXT example.com\PYGZdq{}
example.com descriptive text \PYGZdq{}v=spf1 a:mail.example.com \PYGZhy{}all\PYGZdq{}
\end{sphinxVerbatim}

\sphinxAtStartPar
Note that it can take a while until a DNS entry is propagated.


\subsection{Set \sphinxstyleliteralintitle{\sphinxupquote{DKIM}} signature}
\label{\detokenize{quickstart:set-dkim-signature}}
\sphinxAtStartPar
On your server, the \sphinxcode{\sphinxupquote{rspamd}} systemd service generated a file
containing your DKIM public key in the file
\sphinxcode{\sphinxupquote{/var/dkim/example.com.mail.txt}}. The content of this file looks
like

\begin{sphinxVerbatim}[commandchars=\\\{\}]
\PYG{n}{mail}\PYG{o}{.}\PYG{n}{\PYGZus{}domainkey}      \PYG{n}{IN}      \PYG{n}{TXT}     \PYG{p}{(} \PYG{l+s+s2}{\PYGZdq{}}\PYG{l+s+s2}{v=DKIM1; k=rsa; }\PYG{l+s+s2}{\PYGZdq{}}
   \PYG{l+s+s2}{\PYGZdq{}}\PYG{l+s+s2}{p=\PYGZlt{}really\PYGZhy{}long\PYGZhy{}key\PYGZgt{}}\PYG{l+s+s2}{\PYGZdq{}} \PYG{p}{)}  \PYG{p}{;} \PYG{o}{\PYGZhy{}}\PYG{o}{\PYGZhy{}}\PYG{o}{\PYGZhy{}}\PYG{o}{\PYGZhy{}}\PYG{o}{\PYGZhy{}} \PYG{n}{DKIM} \PYG{n}{key} \PYG{n}{mail} \PYG{k}{for} \PYG{n}{nixos}\PYG{o}{.}\PYG{n}{org}
\end{sphinxVerbatim}

\sphinxAtStartPar
where \sphinxcode{\sphinxupquote{really\sphinxhyphen{}long\sphinxhyphen{}key}} is your public key.

\sphinxAtStartPar
Based on the content of this file, we can add a \sphinxcode{\sphinxupquote{DKIM}} record to the
domain \sphinxcode{\sphinxupquote{example.com}}.


\begin{savenotes}\sphinxattablestart
\sphinxthistablewithglobalstyle
\centering
\begin{tabulary}{\linewidth}[t]{TTTT}
\sphinxtoprule
\sphinxstyletheadfamily 
\sphinxAtStartPar
Name (Subdomain)
&\sphinxstyletheadfamily 
\sphinxAtStartPar
TTL
&\sphinxstyletheadfamily 
\sphinxAtStartPar
Type
&\sphinxstyletheadfamily 
\sphinxAtStartPar
Value
\\
\sphinxmidrule
\sphinxtableatstartofbodyhook
\sphinxAtStartPar
mail.\_domainkey.example.com
&
\sphinxAtStartPar
10800
&
\sphinxAtStartPar
TXT
&
\sphinxAtStartPar
\sphinxcode{\sphinxupquote{v=DKIM1; k=rsa; s=email; p=\textless{}really\sphinxhyphen{}long\sphinxhyphen{}key\textgreater{}}}
\\
\sphinxbottomrule
\end{tabulary}
\sphinxtableafterendhook\par
\sphinxattableend\end{savenotes}

\sphinxAtStartPar
You can check this with

\begin{sphinxVerbatim}[commandchars=\\\{\}]
\PYGZdl{} nix\PYGZhy{}shell \PYGZhy{}p bind \PYGZhy{}\PYGZhy{}command \PYGZdq{}host \PYGZhy{}t txt mail.\PYGZus{}domainkey.example.com\PYGZdq{}
mail.\PYGZus{}domainkey.example.com descriptive text \PYGZdq{}v=DKIM1;p=\PYGZlt{}really\PYGZhy{}long\PYGZhy{}key\PYGZgt{}\PYGZdq{}
\end{sphinxVerbatim}

\sphinxAtStartPar
Note that it can take a while until a DNS entry is propagated.


\subsection{Set a \sphinxstyleliteralintitle{\sphinxupquote{DMARC}} record}
\label{\detokenize{quickstart:set-a-dmarc-record}}
\sphinxAtStartPar
Add a \sphinxcode{\sphinxupquote{DMARC}} record to the domain \sphinxcode{\sphinxupquote{example.com}}.


\begin{savenotes}\sphinxattablestart
\sphinxthistablewithglobalstyle
\centering
\begin{tabulary}{\linewidth}[t]{TTTT}
\sphinxtoprule
\sphinxstyletheadfamily 
\sphinxAtStartPar
Name (Subdomain)
&\sphinxstyletheadfamily 
\sphinxAtStartPar
TTL
&\sphinxstyletheadfamily 
\sphinxAtStartPar
Type
&\sphinxstyletheadfamily 
\sphinxAtStartPar
Value
\\
\sphinxmidrule
\sphinxtableatstartofbodyhook
\sphinxAtStartPar
\_dmarc.example.com
&
\sphinxAtStartPar
10800
&
\sphinxAtStartPar
TXT
&
\sphinxAtStartPar
\sphinxcode{\sphinxupquote{v=DMARC1; p=none}}
\\
\sphinxbottomrule
\end{tabulary}
\sphinxtableafterendhook\par
\sphinxattableend\end{savenotes}

\sphinxAtStartPar
You can check this with

\begin{sphinxVerbatim}[commandchars=\\\{\}]
\PYGZdl{} nix\PYGZhy{}shell \PYGZhy{}p bind \PYGZhy{}\PYGZhy{}command \PYGZdq{}host \PYGZhy{}t TXT \PYGZus{}dmarc.example.com\PYGZdq{}
\PYGZus{}dmarc.example.com descriptive text \PYGZdq{}v=DMARC1; p=none\PYGZdq{}
\end{sphinxVerbatim}

\sphinxAtStartPar
Note that it can take a while until a DNS entry is propagated.


\section{Test your Setup}
\label{\detokenize{quickstart:test-your-setup}}
\sphinxAtStartPar
Write an email to your aunt (who has been waiting for your reply far too
long), and sign up for some of the finest newsletters the Internet has.
Maybe you want to sign up for the \sphinxhref{https://www.freelists.org/list/snm}{SNM Announcement
List}?

\sphinxAtStartPar
Besides that, you can send an email to
\sphinxhref{https://www.mail-tester.com/}{mail\sphinxhyphen{}tester.com} and see how you
score, and let \sphinxhref{http://mxtoolbox.com/}{mxtoolbox.com} take a look at
your setup, but if you followed the steps closely then everything should
be awesome!

\sphinxstepscope


\chapter{Nix Flakes}
\label{\detokenize{flake:nix-flakes}}\label{\detokenize{flake::doc}}
\sphinxAtStartPar
If you\textquotesingle{}re using \sphinxhref{https://wiki.nixos.org/wiki/Flakes}{flakes}, you can use
the following minimal \sphinxcode{\sphinxupquote{flake.nix}} as an example:

\begin{sphinxVerbatim}[commandchars=\\\{\}]
\PYG{p}{\PYGZob{}}
   \PYG{l+s+ss}{description} \PYG{o}{=} \PYG{l+s+s2}{\PYGZdq{}}\PYG{l+s+s2}{NixOS configuration}\PYG{l+s+s2}{\PYGZdq{}}\PYG{p}{;}
   inputs\PYG{o}{.}ghost\PYG{o}{.}\PYG{l+s+ss}{url} \PYG{o}{=} \PYG{l+s+s2}{\PYGZdq{}}\PYG{l+s+s2}{github:QompassAI/ghost/nixos\PYGZhy{}20.09}\PYG{l+s+s2}{\PYGZdq{}}\PYG{p}{;}
   \PYG{l+s+ss}{outputs} \PYG{o}{=} \PYG{p}{\PYGZob{}}self\PYG{p}{,} nixpkgs\PYG{p}{,} ghost\PYG{p}{\PYGZcb{}}\PYG{p}{:} \PYG{p}{\PYGZob{}}
     \PYG{l+s+ss}{nixosConfigurations} \PYG{o}{=} \PYG{p}{\PYGZob{}}
       \PYG{l+s+ss}{hostname} \PYG{o}{=} nixpkgs\PYG{o}{.}lib\PYG{o}{.}nixosSystem \PYG{p}{\PYGZob{}}
         \PYG{l+s+ss}{system} \PYG{o}{=} \PYG{l+s+s2}{\PYGZdq{}}\PYG{l+s+s2}{x86\PYGZus{}64\PYGZhy{}linux}\PYG{l+s+s2}{\PYGZdq{}}\PYG{p}{;}
         \PYG{l+s+ss}{modules} \PYG{o}{=} \PYG{p}{[}
           ghost\PYG{o}{.}nixosModule
           \PYG{p}{\PYGZob{}}
             \PYG{l+s+ss}{ghost} \PYG{o}{=} \PYG{p}{\PYGZob{}}
               \PYG{l+s+ss}{enable} \PYG{o}{=} \PYG{n+no}{true}\PYG{p}{;}
               \PYG{c+c1}{\PYGZsh{} ...}
             \PYG{p}{\PYGZcb{}}\PYG{p}{;}
       \PYG{p}{\PYGZcb{}}
         \PYG{p}{]}\PYG{p}{;}
     \PYG{p}{\PYGZcb{}}\PYG{p}{;}
    \PYG{p}{\PYGZcb{}}\PYG{p}{;}
  \PYG{p}{\PYGZcb{}}\PYG{p}{;}
\PYG{p}{\PYGZcb{}}
\end{sphinxVerbatim}


\chapter{Indices and tables}
\label{\detokenize{index:indices-and-tables}}\begin{itemize}
\item {} 
\sphinxAtStartPar
\DUrole{xref}{\DUrole{std}{\DUrole{std-ref}{genindex}}}

\item {} 
\sphinxAtStartPar
\DUrole{xref}{\DUrole{std}{\DUrole{std-ref}{modindex}}}

\item {} 
\sphinxAtStartPar
\DUrole{xref}{\DUrole{std}{\DUrole{std-ref}{search}}}

\end{itemize}



\renewcommand{\indexname}{Index}
\printindex
\end{document}